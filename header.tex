



\usepackage{fontenc}
\usepackage{booktabs}


% Captions schöner machen.
\usepackage[
  labelfont=bf,        % Tabelle x: Abbildung y: ist jetzt fett
  font=footnotesize,          % Schrift etwas kleiner als Dokument
  width=0.9\textwidth, % maximale Breite einer Caption schmaler
  format=hang,
  justification=raggedright,
]{caption}
\usepackage[english]{babel}
\usepackage{csquotes}
% unverzichtbare Mathe-Befehle
\usepackage{amsmath}
% viele Mathe-Symbole
\usepackage{amssymb}
% Erweiterungen für amsmath
\usepackage{mathtools}
% subfigure, subtable, subref
\usepackage{subcaption}
\usepackage{physics}
% Grafiken können eingebunden werden
\usepackage{graphicx}
% größere Variation von Dateinamen möglich
\usepackage{grffile}
% Verbesserungen am Schriftbild
\usepackage{microtype}

\usepackage{xargs}

% % Hyperlinks im Dokument
% \usepackage[
%   german,
%   unicode,        % Unicode in PDF-Attributen erlauben
%   pdfusetitle,    % Titel, Autoren und Datum als PDF-Attribute
%   pdfcreator={},  % ┐ PDF-Attribute säubern
%   pdfproducer={}, % ┘
% ]{hyperref}
% % erweiterte Bookmarks im PDF
% \usepackage{bookmark}

%Trennung von Wörtern mit Strichen
\usepackage[shortcuts]{extdash}

\usepackage{catchfile}
\usepackage{xfrac}

% \usepackage[%
%   autocite    = superscript,
%   backend     = biber,
%   sortcites   = true,
%   style       = numeric,
%   ]{biblatex}
% \DeclareCiteCommand{\supercite}[\mkbibsuperscript]{
%   \iffieldundef{prenote}
%   {}
%   {\BibliographyWarning{Ignoring prenote argument}}%
%   \iffieldundef{postnote}
%   {}
%   {}
% }
% {\bibopenbracket%
%   \usebibmacro{citeindex}%
%   \usebibmacro{cite}%
%   \usebibmacro{postnote}%
%   \bibclosebracket}
% {\supercitedelim}
% {}
% \newcommandx*\citeauthyear[3][2=(,3=)]{\textsuperscript{#2\citeauthor{#1}~\citeyear{#1}#3}}
% \newcommandx*\twociteauthyear[4][3=(,4=)]{\textsuperscript{#3\citeauthor{#1}~\citeyear{#1},~\citeauthor{#2}~\citeyear{#2}#4}}
% \newcommand{\mycaption}[1]{
  
% \stepcounter{figure}\footnotesize\textbf{Figure \arabic{figure}:} #1\normalsize
% }
\newcommand\sub[1]{\ensuremath{_{\mathrm{#1}}}}
%\newcommmand\arrowitem{\item[$\rightarrow$]}
% \addbibresource{bibliography.bib}
% ç
\usepackage[
  math-style=ISO,    % ┐
  bold-style=ISO,    % │
  sans-style=italic, % │ ISO-Standard folgen
  nabla=upright,     % │
  partial=upright,   % ┘
  warnings-off={           % ┐
    mathtools-colon,       % │ unnötige Warnungen ausschalten
    mathtools-overbracket, % │
  },                       % ┘
]{unicode-math}
\usepackage{multicol}
\usepackage[
  locale=US,                   % deutsche Einstellungen
  separate-uncertainty=true,   % immer Fehler mit \pm
  per-mode=symbol-or-fraction, % / in inline math, fraction in display math
]{siunitx}

\usepackage{xfrac}
\sisetup{per-mode = fraction, separate-uncertainty,quotient-mode=fraction,math-micro=\text{µ},text-micro=µ,detect-family,detect-display-math = true}
%\renewcommand{\boxed}[1]{\text{\fboxsep=.2em\fbox{\m@th$\displaystyle#1$}}}
\usepackage{capt-of}

% \usepackage{textcomp}
% \usepackage{mathptmx}
% \usepackage{helvet}
% \setbeamertemplate{navigation symbols}{}
% \setbeamercovered{transparent}
% \usefonttheme[onlymath]{serif}
\usepackage{xfrac}
\usepackage{times}
\usepackage{lipsum}
\setmathfont{XITSMath-Regular}